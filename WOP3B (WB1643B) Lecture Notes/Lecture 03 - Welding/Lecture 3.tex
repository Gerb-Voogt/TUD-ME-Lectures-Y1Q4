\documentclass[11pt, a4paper]{article}

\usepackage{graphicx}
\usepackage[a4paper,top=3cm,bottom=2cm,left=2cm,right=2cm,marginparwidth=1.75cm]{geometry}
\usepackage[english]{babel}
\usepackage[utf8x]{inputenc}
\usepackage{subfig}
\usepackage{amsmath}
\usepackage{amssymb}
\usepackage{mhchem}

\graphicspath{ {./images} }
\newcommand*{\qed}{\hfill\ensuremath{\quad\square}}%
\newcommand*{\rad}{\ensuremath{\,\text{rad}}}
\newcommand*{\R}{\ensuremath{\mathbb{R}}}

\makeatletter
\renewcommand*\env@matrix[1][*\c@MaxMatrixCols c]{%
  \hskip -\arraycolsep
  \let\@ifnextchar\new@ifnextchar
  \array{#1}}
\makeatother

\newtheorem{theorem}{Theorem}

%------------------------------------------------
%Templates for images and figures
% \begin{figure}[h]
%   \centering
%   \subfloat[caption 1]{{\includegraphics[width=30mm]{images/placeholder.png}}}%
%   \qquad
%   \subfloat[caption 2]{{\includegraphics[width=30mm]{images/placeholder.png}}}%
%   \caption{Description}
% \end{figure}

% \begin{figure}[h]
%   \centerline{\includegraphics[width=50mm]{images/placeholder.png}}
%   \caption{Description}
% \end{figure}
%-----------------------------------------------

\begin{document}
\setcounter{equation}{0}
\setcounter{section}{2}
\section{WOP3B Lecture 3: Welding (04/05/2020)}


\subsection{Introduction to welding, brazing and soldering}
Welding usually refers to the joining material through local melting of the material by using heat. The locally melted parts are allowed to flow together, usually with added filler material, and cool down causing fusion of the materials. The 3 main types of welding are:
\begin{itemize}
  \item Arcwelding, using plasma made with high electric current and gas
  \item Gas welding, using a gas as fuel supplying heat (usually oxyfuels)
  \item Resistance welding, using the resistance of the material and high electric currents
\end{itemize}
Brazing is a process comparable to welding. The main difference between welding and brazing is that welding melts the base metal that's being welded while brazing doesn't. Brazing is alot like soldering but at higher temperatures. Using different filler metals such as silver alloys, copper alloys or aluminum alloys. Soldering generally only uses tin alloys or lead alloys as filler material depending on the goal.


\subsection{Weld quality and material properties}
Welding adds a high amount of localized heat. Heat generally changes material properties, thus it's only natural to conclude that welding may have large implications for the local material properties. The quality of the weld itself is also highly dependend on the process. Examples of process paramaters which influence weld quality are heat transferred during welding, heat transport through convection (during and after welding), solidification properties of the material and weldpuddle interactions such as difference in density or interaction with electro-magnetic forces.\\
Welding affects the properties of the surrounding material because of the large amount of heat. The mechanical properties of a material are determined by the microstructure of the material, which changes due to heat. This local change in microstructure can affect properties such as local tensile strength, local ductility of the material, local toughness and crack sensetivity. This means mechanical properties in and around a weld are different from the base metal which wass welded. This has some important implications when considering part design. On top of this welding can cause local deformations and residue stresses due to the center of the weld cooling slower then the surrounding area.


\subsection{The Heat Affected Zone (HAZ)}
Concentrated amounts of local heat will dissapate through the material causign material around the weld to heat up as well. This zone is referred to as the heat affected zone. This zone is defined by the perimeter around the weld where changes to the microstructure happen. As discussed previously these local changes in microstructure also cause local chnages in material properties. For steel the heat affected zone has 3 main parts. The weld itself where grains tend to be very long and large and grow in the direction of the maximum temperature gradient (almost always towards the weld direction). Just outside the weld is a perimeter with very coarse grains which tend to grow towards the weld metal. Outside that area is an area with fine grains which is the outer most part of the HAZ.


\subsection{The math describing heat development over time}
The full description of heat transfer during welding is given by the following equation. This equations is an energy sum describing the amount of energy going in and out of the weld:
\begin{equation}
  \rho C_p \frac{\partial T}{\partial t} = \kappa \nabla^2 T - \rho C_p (\vec{v}\cdot \nabla)T - U + \vec{j} \cdot \vec{E} + \phi
\end{equation}
Where $\kappa \nabla^2 T$ is heat conduction through the material, $\rho C_p (\vec{v} \cdot \nabla)T$ is the loss of heat through convection, $U$ is the loss of heat through radiation, $\vec{j}\cdot \vec{E}$ is the electrical energy added to the weld and $\phi$ is the dissapation of heat. For this partial differential equation (PDE) to be analytically solvable a ton of simplification need to be made. When $\alpha = \frac{\kappa}{\rho C_p}$ then we are left with the solvable PDE, the heat equation:
\begin{equation}
  \frac{\partial T}{\partial t} = \alpha \nabla^2 T
\end{equation}
This equations tells us that the change of heat over time is dependend on some constant $\alpha$ which depends on material and the spatial second derrivatives (laplacian) of the temperature. The laplaciaan can be thought of as the divergence of the gradient, but the most important fact for now is that the cahnge of heat over time depends on the temperature gradient.\\
Applying the heat equation to a weld, as mentioned before, simplifies the problem to the point of meaninglessness since it doesn't \underline{really} tell us anything about the weld anymore at that point. Instead generally numerical methods such as finite element analysis are applied to systems like this since they are much more accurate.


\subsection{External welding defects}
\underline{Underfilling:} When welding sometimes the weld is not completely filled compared to the surrounding material. This usually happens because of welding at high speeds, not using enough power or using too little filler material.
\newline
\underline{Overfilling:} Conversly welds can also be overfilled rather then underfilled. When this happens the weld extends above the surrounding material. It usually happens when welding too slow or when using too much filler material.
\newline
\underline{Root or toe cracks:} It's possible for small crevices and cracks to form at either the toe of root of a weld. It usually happens when using too much power, using an incorrect electrode angle when welding or when the distance between electro and the workpiece is too large. It might also occur due to surface contamination.
\newline
\underline{Undercutting:} Because of thermodynamics a weld solidifies and by extension shrinks in the direction of the largest temperature gradient. This can cause the forming of an undercut at the root of the weld. This can be prevented by adding less heat when doing a second pass over the weld.
\newline
\underline{Linear missalignment:} Welds can cause parts to misalign due to either thermal shrinking or bad craftsmanship. Advoid this by taking good care of the weld, using proper clamping when welding and look for a better welder when it does happen.
\newline
\underline{Cold lapping:} Sometimes a weld is in contact with but not fused to the mother material. This is referred to as cold lapping. It's usually caused by bad welding technique or bad choice of mother material. Not all metals are weldable (7075 \ce{Al} is not for example.)


\subsection{Internal welding defects}
\underline{porosity:} When welding small pores may be created inside of the weld. It's usually a result of gas forming in the weld. This may happen due to the surface being contaminated for example. Shrinking or using too little protective gas may also be causes for porosity.
\newline
\underline{Slag inclusion:} Slag can get stuck inside of a weld during the welding process.
\newline
\underline{Lack of fusion:} The weld metal may not completely fuse with the mother metal. Can have many different causes, such as high weld speed, bad weld design, insufficient pre-heating, etc.
\newline


\subsection{Forming of cracks}
\underline{Cracking during solidification:} Usually happens at the center of the weld. Outside of the weld in contact the mother metal cools down and by extension solidifies quicker then the center of the weld. This can create a crevice or brittle fracture at the center of the weld. Welds with a high depth:width ratio are especially susceptible to this.
\newline
\underline{Liquation cracking:} Cracking due to local difference in morphology. Example is \ce{Mn} not being homogeneously distributed through a piece of steel, but rather having a high local concentration in 1 or a few grains. This high local concentration of \ce{Mn} can cause the material locally melt at a lower temperature. Alloys with large solidification trajectory also have a higher tendency to undergo liquation cracking.
\newline
\underline{Hydrogen embrittlement:} Has been discussed before in the lecture on fractures. Caused by hydrogen being present in the weld. Causes a sudden cold, brittle fracture.


\subsection{Inspection of welds}
Short summation of common weld inspection methods:
\begin{itemize}
  \item Visual, looking at the weld for defects with your eyes
  \item Penetrant, visualizing cracks and possible pores with a high contrast fluid and UV-light
  \item Magnetic, visualizing cracks by getting coloured magnetic particles caught in the crack
  \item Radiographic, x-raying a weld
  \item Ultrasonic, Using echo location on a weld
\end{itemize}


\subsection{Micro and nano defects}
No weld is completely without defects. Micro defects and nano defects such as very small pores and cracks may (and will) occur. These micro cracks may grow into macro cracks under an applied load.


\end{document}