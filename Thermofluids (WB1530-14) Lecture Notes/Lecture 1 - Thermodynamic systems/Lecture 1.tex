\documentclass[11pt, a4paper]{article}

\usepackage{graphicx}
\usepackage[a4paper,top=3cm,bottom=2cm,left=2cm,right=2cm,marginparwidth=1.75cm]{geometry}
\usepackage[english]{babel}
\usepackage[utf8x]{inputenc}
\usepackage{subfig}
\usepackage{amsmath}
\usepackage{amssymb}
\usepackage{array}

\graphicspath{ {./images} }
\newcommand*{\qed}{\hfill\ensuremath{\quad\square}}%
\newcommand*{\rad}{\ensuremath{\,\text{rad}}}
\newcommand*{\R}{\ensuremath{\mathbb{R}}}

\makeatletter
\renewcommand*\env@matrix[1][*\c@MaxMatrixCols c]{%
  \hskip -\arraycolsep
  \let\@ifnextchar\new@ifnextchar
  \array{#1}}
\makeatother

\newtheorem{theorem}{Theorem}

%------------------------------------------------
%Templates for images and figures
% \begin{figure}[h]
%   \centering
%   \subfloat[caption 1]{{\includegraphics[width=30mm]{images/placeholder.png}}}%
%   \qquad
%   \subfloat[caption 2]{{\includegraphics[width=30mm]{images/placeholder.png}}}%
%   \caption{Description}
% \end{figure}

% \begin{figure}[h]
%   \centerline{\includegraphics[width=50mm]{images/placeholder.png}}
%   \caption{Description}
% \end{figure}
%-----------------------------------------------

\begin{document}
\section{Thermofluids Lecture 1: Thermodynamic systems (20/04/2020)}
\subsection{Units used in the course}
Thermodynamics is the study of heat and energy transfer. There are a number of units which will commonly be found when studying thermodynamics. These are:
\begin{itemize}
  \item Energy: $J = Nm = Ws$
  \item Power: $W = J/s$
  \item Pressure: $Pa = N/m^2$
  \item Mass: $kg$
  \item Temperature: $K$ or $^\circ C$ (Kelvin is preferred)
\end{itemize}
Some other common important units are:
\begin{itemize}
  \item $1 Bar = 10^5 Pa = 100 kPa$
  \item $1 kWh = 3600 kJ$
  \item $1 kCal = 4.18 kJ$
\end{itemize}

\subsection{Definitions of terms used in the course}
\begin{itemize}
  \item \underline{System:}
\end{itemize}
A system is all the matter that is considered when analyzing a given problem. A system is defined by it's boundary. Systems can be either open or closed. Open or closed refers to whether mass can pass throught the boundary of a system. Thus for an open system matter can pass through the system boundary. For closed systems this is not the case.\\

\begin{table}[h!]
  \begin{center}
    \caption{The flows of work heat and mass for open and closed systems}
    \label{tab:table1}
    \begin{tabular}{l|c|c}
      \textbf{ } & \textbf{Closed} & \textbf{Open}\\
      \hline
      diabatic & \includegraphics[ width=30mm]{images/Closed_diabatic.png}
               & \includegraphics[ width=30mm]{images/Open_diabatic.png}\\
      
      adiabatic&\qquad \enspace \includegraphics[ width=20mm]{images/Closed_adiabatic.png}
               &\includegraphics[ width=30mm]{images/Open_adiabatic.png}\\

      isolated &\includegraphics[ width=10mm]{images/isolated.png} & \\
    \end{tabular}
  \end{center}
\end{table}

\begin{itemize}
  \item \underline{continuum:}
\end{itemize}
In thermodynamics matter in considered to be a continuum. This means all matter is observed from a purely macroscopic scale. This means that the density at a single point can be defined as follows:
\begin{equation}
  \rho = \lim_{V \to V'}(\frac{m}{V})
\end{equation}
Where $V'$ is the smallest value for which the ratio $\frac{m}{V}$ excsits.\\

\begin{itemize}
  \item \underline{Thermodynamic equilibrium:}
\end{itemize}
An isolated system will eventually reach a time independent state. This final state is referred to as thermodynamic equilibrium. Classical thermodynamics only considers equilibrium states, and not how these states are reached. The analysis of how these states are reached is often part of statistical mechanics.\\

\begin{itemize}
  \item \underline{State variables or properties:}
\end{itemize}
The state of a system is defined by a small number of variables called the state variables. An example of this is the displacement and velocity of a system in dynamics. A property of a system is independent of the way a state is reached. The magnitude of a change in property only depends on the initial and final states of a system.\\

\begin{itemize}
  \item \underline{intensive, extensive and specific properties:}
\end{itemize}
\textbf{Intensive} properties do not change when the system is changed in size (e.g. Temperature and pressure). \textbf{Extensive} properties change proportional to the size of the system (e.g. mass, volume, energy). Extensive properties can be turned into intensive properties if described per unit of mass (or kmol). These are called \textbf{specific} properties. An example of this is specific volume:
\begin{equation}
  v = \frac{V}{m},\; \bar{v}=vM
\end{equation}
Where:
\begin{flalign*}
  &v = \text{specific volume } [m^3/kg] &&\\
  &\bar{v} = \text{specific volume } [m^3/kmol]\\
  &V = \text{Volume } [m^3]\\
  &m = \text{mass } [kg]\\
  &M = \text{Molar mass } [kg/kmol] 
\end{flalign*}\\

\begin{itemize}
  \item \underline{Thermodynamic process:}
\end{itemize}
A process is defined as a change of state in a system. In classical thermodynamics we only consider the change from one equilibrium state to another equilibrium state. When the initial and final state in a system are identical the process is a cylce.

\begin{itemize}
  \item \underline{Quasi-static process:}
\end{itemize}
A process is considered to be quasi-static if it is reversible. These processes are generally very slow (less then the speed of sound).
\end{document}